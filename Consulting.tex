\documentclass[a4paper,11pt]{article}
\usepackage{CJK}
\usepackage{amsmath,amsfonts,bm}
\usepackage[top=1in, bottom=1in, left=1.25in, right=1.25in]{geometry}
\usepackage{graphicx,epstopdf}
\usepackage{float}
\date{}


% define the title
\author{Bin Zhuo}
\title{\textbf{Homework Assignment  for\\ ST 507: Attandence at Consulting}}
\begin{document}
% generates the title
\maketitle
% insert the table of contents
\vskip 1cm
\newcommand{\tabincell}[2]{\begin{tabular}{@{}#1@{}}#2\end{tabular}}

\section*{Chapter 2}

\subsection*{Exercise 2.3}
\emph{\indent Identify qualities that you would like to have as an ideal statistical consultant. Privately, access yourself on the qualities. Choose one that you feel would be relatively easy for you to attain or improve and develop a plan for doing this.}
\\
\begin{itemize}
  \item A broad base of statistical and subject-material knowledge
  \item Good oral and written communication skills
  \item The ability to learn quickly what you don't know
  \item Get highly involved in the solution of company problems
  \item produce high-quality work in a timely fashion
  \item Good statistical programming skills.
\end{itemize}
Currently, I'm learning more statistical theory and methods. There is a long way to improve my speaking and writing skills. One thing I can do now is to write code whenever necessary to solve problems I come across in my everyday study.
\subsection*{Exercise 2.7}
\emph{\indent Read through the comments from dissatisfied and satisfied clients in Exhibits 2.2 and 2.3. For each comment identify the dimension of customer satisfaction from Exhibit 2.4 that are involved.}
\begin{center}
\begin{tabular}{|l|l|}
  \hline
  \footnotesize{Availability of support} & \footnotesize{the degree to which the customer can contact the provider}\\ \hline
  \footnotesize{Responsiveness of support} &\footnotesize{the degree to which the provider react promptly to the customer}\\ \hline
 \footnotesize{Timeliness of support} &\footnotesize{\tabincell{l}{the degree to which the job is accomplished within the\\ customer's stated time frame and/or within the negotiated time frame}}\\ \hline
\footnotesize{Completeness of support} &\footnotesize{the degree to which the total job is finished}\\ \hline
\footnotesize{Pleasantness of support} & \footnotesize{ \tabincell{l}{the degree to which the provider uses suitable professional \\behavior and manners while working with the customer}}\\ \hline
\end{tabular}
\end{center}
The satisfied clients:
\begin{enumerate}
         \item Responsiveness of support; Completeness of support.
         \item Completeness of support; Availability of support; Pleasantness of support.
         \item Completeness of support; Pleasantness of support.
         \item Completeness of support; Pleasantness of support.
         \item Completeness of support
         \item Completeness of support
         \item Completeness of support
\end{enumerate}
The dissatisfied clients:
\begin{enumerate}
  \item[8]: Completeness of support; Availability of support.
  \item[9]: Timeliness of support.
  \item[10]: Timeliness of support; Responsiveness of support.
  \item[11]: Completeness of support.
  \item[12]: Completeness of support.
\end{enumerate}
\subsection*{Exercise 3}
\emph{\indent Of the seven types of intelligence, which categories most describe you? which categories describe the types of people with which you have difficulty communicating?}\\
\\
 I would describe myself as Logical/Mathematical intelligence. In my college I was trained to perform logical thinking. I think it would be people with Body/Kinesthetic intelligence, or Intrapersonal intelligence.


 \section*{Chapter 3}
 \subsection*{Problem 1}
 \emph{\indent Describe five components (such as eye contact) of non-verbal communication. For each component, give an example of a behavior that might be negatively interpreted.}
 \begin{paragraph}
  {The setting.}The layout of a meeting room communicates messages about you to your client. Lighting, color, and the type and arrangement of office furniture all affect people's perception and ability to work on different types of task. For example, when your client come to your office, you speak across your desk, which might put yourself in a dominant position with respect to your client. That is a negative version.
 \end{paragraph}
 \begin{paragraph}
   {The greeting.}The way you and your client greet each other can influence the comfort of the upcoming discussion, and it might result in a quick emotional reaction. For example the client comes in, the consultant says hello without looking at the client. That would certainly conveys an impression of disrespect.
 \end{paragraph}
 \begin{paragraph}
   {Eye contact.} Eye contact can be used to convey your interest and understanding to the client. If your client is making more or less eye contact with you than you expect him to, you might may begin to develop a negative reaction to him without knowing why. However, if your client is making more eye contact than you expect, you might be feeling if he is being overly aggressive.
 \end{paragraph}
\begin{paragraph}
  {Interpersonal Distance} Interpersonal distance is the distance that people will choose to stand or sit from each other, which arise from people's need for personal space. The distance is defined both by cultural norms and by the natural relationship between two people. For example, if a consultant sits right beside the client, the client might feel that her personal space has been invaded. That would make her uncomfortable or even threatened.
\end{paragraph}
\begin{paragraph}
  {Body Posture} An open posture, which has you leaning slightly towards the client, will help to convey your interest to your client. A closed posture, with your arms and/or legs crossed and your body leaning away from the client, can convey disinterest. What's more, appropriate body posture varies between cultures. Usually, a closed posture will result in a negative impression to other people.
\end{paragraph}

\subsection*{Problem 2}
\emph{\indent What difficulties do you see with non-verbal communication in consulting?}\\

The presenter  ignores or does not notice that non-verbal behavior of other member in that group. For example, Ann kept speaking even if Dr. Klerk grabbed her arm, interrupted her, and proceeded to disagree with her. Other members around the table looked uncomfortable. The end result is that the meeting was in trouble.

\subsection*{Problem 3}
\emph{\indent What steps can you take to notice problems in cross-cultural consulting situation?}\\

1. I should know some background about my client, e.g. greeting in his culture, personal space when talking.
2. Try to behave following his custom.
3. If such information is not available, respect his communication style if I do not feel offended.

\subsection*{Problem 4}
\emph{\indent Fill out Exercies 3.1 pg.98.}
\begin{center}
  \begin{tabular}{|l|l|}\hline\hline
   \multicolumn{2}{c} {Your reference culture, county or region}\\ \hline\hline
   \footnotesize{How late you can be for the first meeting without offending the client} &\footnotesize{five minutes}\\ \hline
   \footnotesize{How late was the client can be for the first meeting without offending you} & \footnotesize{10 minutes}\\ \hline
   \footnotesize{The way you greet the client} &\tabincell{c}{\footnotesize{\tabincell{l}{Shaking hands and\\ maybe a cup of tea}}}\\ \hline
   \footnotesize{\tabincell{l}{The distance you and your client sit\\ apart from each other at a table}} &\footnotesize{ at least 0.5m }\\ \hline
   \footnotesize{The amount of eye contact that you make with your client} &\footnotesize{\tabincell{l}{ as often as possible, \\but don't stare at your client}} \\ \hline
   \footnotesize{The amount of eye contact that your client makes with you} & \footnotesize{\tabincell{l}{as often as possible, \\ but don't stare at your client}}\\ \hline
   \footnotesize{\tabincell{l}{How you indicate agreement or understanding\\ with what your client has said}} & \footnotesize{\tabincell{l}{Nodding your head and\\ clapping your hands}} \\ \hline
  \footnotesize{ What form (if any) of touching is acceptable }& \footnotesize{usually no touching at all. }\\ \hline
  \end{tabular}
\end{center}

\section*{Chapter 4}
\subsection*{Problem 1}
\emph{\indent Define the four dimensions of communication given in chapter 4. For each dimensions, privately reflect on your own style.}
\begin{paragraph}
  {Phasing} refers to the conventions people have for when to talk about certain topic. In a business or statistical consulting conversation, these conventions about phasing affect the timing and duration of small talk and business conversation. Personally, I definitely want to arrange a meeting in detail with my clients first before the meeting being held.
\end{paragraph}

\begin{paragraph}
  {Sequencing} refers to the way people prefer to structure the topics within a conversation. People on one extreme of the continuum prefer to start with one topic and finish discussing it completely before proceeding to the next topic. People on the other extreme prefer to discuss several topics at once, branching off on tangents and returning more than once to topics in a circular or spiraling fashion during the conversation. I'm on neither of the two extremes. I would like to discuss one topic in detail, but not necessarily through before moving to the next.
\end{paragraph}

\begin{paragraph}
  {Specificity} refers to the way people prefer to arrange general and specific information about a topic. Some people prefer talking about specific details first and then moving to generalities, others are more comfortable beginning with generalities and moving to specific information. Privately, I prefer outline the generalities of an issue before the specific information, and that helps me keeping on track what we are talking about.
\end{paragraph}

\begin{paragraph}
  {Objectivity} refers the way people use language to convey their ideas. Some people prefer to use precise language to convey their meanings very directly. Others prefer to be more indirect, relying more on context and inference to get their ideas cross. I prefer to use precise language most of the time. However, I do sometimes would be indirect, trying to be not offensive.
\end{paragraph}
\subsection*{Problem 2}
\emph{\indent It is likely that your first contact with a client will be via email. Construct a hypothetical email message using ideas from the book.} \\
\\
Dear Mr. Bryant,\\
\\
It's was great to receive your email. I would certainly be happy to work with you about this project. According to your plan, I'd like to arrange a short meeting with you. Here's the topic I can think of:\\
1. The objective of your project (be specific)\\
2. What kind of data is available\\
3. Analysis plan and conclusions that are acceptable\\
4. Deadline for this project.
\\
Is 2pm  Monday convenient for you? Please feel free to call or email me if you have any additional questions.

\subsection*{Problem 3}
\emph{\indent Describe several important differences in the interpersonal dynamics of consulting in a one-on-one meeting as compared to a group meeting. What are some differences in effective communication styles for one-on-one versus group meetings?}\\

In a one-on-one meeting, the consultant will have good opportunity to learn about and adapt to his client's preferred style of communication, be more flexible about the discussion,  as well as shapes the conversation according to the mutual preferences. While a large group meeting will probably have a leader, and this person will be more constrained to move the discussion along in a way that benefits the entire group. Group meeting is also more formal than a one-on-one meeting.

\section*{Chapter 5}
\subsection*{Problem 1}
\emph{\indent What are "errors of the third kind", and how can you reduce the likelihood of making them?}
\\

"Errors of the third kind" refer to providing the right answer to the wrong problem. You should have a sense of commitment when clients turn to you, possess enough knowledge about the clients' field to provide a good recommendation, and take enough time to discuss the problem thoroughly or to look carefully at the data.
\subsection*{Problem 2}
\emph{\indent What is paraphrasing? When and how is it helpful during statistical consulting?}
\\

Paraphrasing is a restatement of factual information. It allows you to reflect to your client what your understanding is of what he has told you. Summarizing your understanding will help you and your client make shure that you have correctly understood. If you have missed anything, a paraphrase will help the client to correct this misunderstanding.

\subsection*{Problem 3}
\emph{\indent What is an open probe? What is a closed probe? What are the advantages and disadvantages of each?}
\\

An open probe is a question or a comment that prompts the client to discuss a topic more generally. Open questions are broad, general questions that give the client a lot of freedom in responding.
A closed probe is a question or a comment that prompts for a brief response from the client. Closed probes are useful for obtaining specific information. However, an incomplete forced choice question or a not clearly worded question may cause the client to modify her answers to suit the consultant's definition.
\subsection*{Problem 4}
\emph{\indent All of the questions in dialog 2 on page 79 are closed probes. For each question, rephrase the question as an open probe that could be used to obtain the same information.}
\\

1. How did you assign the pigs to the treatments? 2. What's an independent experimental units?
3. How did you feed them? 4. In what order did you observe their behavior?
\subsection*{Problem 5}
\emph{\indent Under what circumstances might a client tell a statistician what the client thinks the statistician wants to hear? }
\\

If the consultant paraphrases what the client just told or asked him, the client would tell him.
\section*{Chapter 6}
\subsection*{Problem 1}
\emph{\indent What does it mean to have a "win-win" outcome in statistical consulting?}
\\

A "win-win" outcome is one in which all parties involved feel that their needs will be met. In this ideal outcome, everyone clearly understands the exchanges involved in the agreement. They all feel that the exchanges are fair. In statistical consulting,  a "win-win" outcome is when you and your client are aligned on the expectations for the key issues that affect the project.
\subsection*{Problem 2}
\emph{\indent Compare and contrast "high context" and "low context" style of negotiation}
\begin{paragraph}
{High context.} 1. Prefers to establish a personal relationship entering into a negotiation. 2. Draw as many inferences surroundings, from nonverbal cues, and from hinted nuances of meanings. 3. Will avoid "haggling" about important issues, instead, will state his position and keep to it. 4. Prefers indirect communication, avoids offending others or causing embarrassment. 5. Prefers to follow previously established conventions.
\end{paragraph}
\begin{paragraph}
  {Low context.} 1. Prefer to start negotiating immediately without establishing a personal relationship. 2. Obtains meanings mainly from verbal discussion. 3. Is willing to "haggle" about most issue, including very important ones. 4. Values accuracy in direct language and will speak very frankly. 5. Does not fell tied to conventions; is receptive to innovation.
\end{paragraph}
\subsection*{Problem 3}
\emph{\indent List as many "intangible benefits" of statistical consulting as you can think of. Privately reflect on which ones are important to you at this stage of your statistical education, and which might be important to you in the future.}
\\

1. The pleasure that the client feel can bring trust, and maybe future cooperation or introducing other clients to you. 2. The successful consulting experience will contribute to your reputation among consulting community. 3. Personal relationship with your client will be enhanced. 4. Statistical knowledge and specific field knowledge you will gain via this consulting experience. 5. The ability to evaluate the cost of similar project. As for me, that I'll gain practical consulting experience will help me to find a good job is rewarding. To develop a nice relationship with my client would be beneficial to my future career.
\subsection*{Problem 4}
\emph{\indent For example 6.2, 6.3, and 6.4 on page 103-104, explain whether the outcome is a "win-win", "lose-lose", "win-lose", or something else. For those that are "win-lose" or "lose-lose", suggest what could have been done to make the outcome a "win-win".}
\\

Example 6.2 is a "win-win", 6.3 is a "loss-loss", and 6.4 is a "win-lose". \\
In example 6.3, Nelson should have asked if Gregoriou is available or if he is interrupting at first call. Realizing this to be not a "quick question", Nelson might have made a schedule for a meeting. Gregoriou was being nice, but he failed to express his reluctance when being interrupted.
\\
\noindent In example 6.4, Professor Guitterez should have considered Valerie's suggestion about including as a co-author of his publication, because she worked out the method that accommodated the flaw.
\section*{Chapter 7}
\subsection*{Problem 1}
\emph{\indent  If a client having difficulty understanding your verbal explanation of a statistical concept, what are alternative approaches you could use to get your message across?}
\\

Visual aids, written resource material, good oral presentation, oral and written directions, and opportunities for hands-on practice. Concrete examples and multimedia approach would also help.

\subsection*{Problem 2}
\emph{\indent What are the differences between visual, auditory, and kinesthetic learning style? Reflet on which learning style best describes you.}
\\

Visual learners need to see information. She will read you technical reports carefully wand will benefit from a well-constructed table, diagram or figure. Auditory learners will learn more from your verbal discussion of a project than from reading a technical report. Kinesthetic learners will prefer to have some hands-on exposure to the technical information you are providing. I would say I may be defined as a visual learner more than the other two.\\


\subsection*{Problem 3}
\emph{\indent Select any statistical term or concept, and provide an explanation that a scientist with no statistical training could understand. You may use diagrams (hand written) graphs in your explanation.}
\begin{paragraph}
{Confidence Interval} is a type of interval estimate of a population parameter and is used to indicate the reliability of an estimate. In a common view, this interval is an interval that captures the true parameter of a population of interest. For example, a 95\% confidence interval for the average weight is (100,150). Suppose we get 100 samples from the population, and we can obtain 100 mean weights. This confidence interval tells us that we would anticipate about 95 sample means are within (100,150). We are 95\% sure that (100,150) will cover the true mean weight of this population.
\end{paragraph}
\end{document}
